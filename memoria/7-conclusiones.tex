Tras detallar en profundidad las mejoras aportadas a la herramienta Scratch4Robots,  este capítulo se ha reservado para ver hasta qué punto se han logrado los objetivos establecidos, para resumir los conocimientos adquiridos durante su desarrollo y para exponer las posibles mejoras que se pueden introducir en trabajos futuros.

\section{Conclusiones}
\label{sec:conclusiones}

El objetivo global de este proyecto era la mejora de la herramienta Scratch4Robots 1.0. Este objetivo se ha alcanzado con éxito a través de una serie de subobjetivos. Estos subobjetivos se dividen en la refactorización de la herramienta, y la extensión en funcionalidad de ésta con el desarrollo de nuevos bloques, una paquetización completa de la herramienta, y la validación experimental con robots ROS.\\

El primer subobjetivo consiste en la refactorización y extensión en funcionalidad. Esto se consigue haciendo modificaciones tanto en el proceso de traducción, en la generación del nodo final ejecutable y en los propios bloques ya existentes en la herramienta. La traducción que aportamos ofrece una mayor capacidad de detección de bloques anidados, además de soportar una mayor cantidad de bloques propios de Scratch. \\

Como parte de esta refactorización, el código generado, ahora es un único nodo ROS programado en Python, con toda la lógica para funcionar de forma autónoma, a este nodo ROS se le aporta la flexibilidad de ser configurado a través de un fichero de configuración, en el que se indican las necesidades de nuestro desarrollo, como pueden ser sensores, actuadores etcétera. \\

Otro de los cambios que aporta esta refactorización es la adaptación a comunicaciones ROS en su totalidad. Se añade el uso de listas para la aceptación parámetros complejos de entrada y salida por parte de los bloques. En cuanto a la extensión de funcionalidades, aportamos nuevos bloques. Como bloque destacado se agrega el bloque perceptivo de detección de objetos de un color determinado, que usa una cámara montada sobre el robot. Se aportan otros bloques de carácter general, como son los bloques matemáticos y lógicos, además de todos los bloques para el manejo de listas. Estos bloques amplían las posibilidades de uso de la herramienta de forma considerable.\\

En segundo lugar se busca una paquetización completa de la herramienta. Esto se consigue de dos formas. La primera y principal, es la generación de un paquete ROS, instalable en forma de binario desde linea de comandos en un sistema operativo Ubuntu. Este paquete además de ser fácilmente instalable, hace uso de todas las funcionalidades que nos aporta el entorno ROS, como por ejemplo la ejecución de nodos pertenecientes a un paquete ya instalado, desde linea de comandos. Esta funcionalidad es en la que nos basamos para lanzar el nodo que realiza la traducción del bloque Scratch al nodo ROS final. Además de esta paquetización ROS, realizamos otra labor de desacople de dependencias en forma de paquetes pip, igualmente instalables desde linea de comandos. Esto aporta una mayor flexibilidad a la herramienta, pudiendo agregar y modificar funcionalidad sin necesidad de modificar el core de la herramienta.\\

Por último queremos hacer una validación experimental de la herramienta con robots puramente ROS. Este subobjetivo se lleva a cabo con la total incorporación de comunicaciones ROS, tanto para Turtlebots como para drones. Estas comunicaciones son validadas con una serie de ejemplos funcionales sobre robots simulados.\\

El objetivo se ha conseguido cumpliendo con los requisitos previos que nos habíamos propuestos. Todo el código se realiza en lenguaje Python. El paquete ROS generado, se trata de un ROS en su versión Kinetic, versión que está enfocada a su uso sobre Ubuntu 16.04, otro de los requisitos que establecíamos. Además no necesitamos de más software, aparte del mencionado anteriormente y el propio Scratch 2.0 en su versión \textit{offline}.


\section{Trabajos Futuros}
\label{sec:trabajos-futuros}

El desarrollo de este trabajo ha generado la versión 2.0 de Scratch4Robots, pero aún quedan muchas cosas que se pueden desarrollar para mejorar la herramienta.\\

El más destacado y de necesaria implementación para que la herramienta se mantenga actualizada, es la integración con Scratch 3.0. Siguiente versión que sale a la luz a principios de 2019. Ayudaría a mantener al día la herramienta con la tecnología del momento. Esto es de vital importancia si buscamos la usabilidad de ésta, una herramienta desactualizada hará que los usuarios se decanten por otras de mayor valor, además de impedirnos el desarrollo de funcionalidades de mayor complejidad por las limitaciones que supone el mantenernos en le versión 2.0 de Scratch. El ser compatibles en un futuro con Scratch 3.0 nos permite crecer tanto cuantitaviamente como cualitativamente, pudiendo ganar más usuarios y además ampliando nuestras funcionalidades, ya que podríamos incorporar funcionalidades como el uso de servicios web que proporciona Scratch 3.0. \\

Otra mejora sería la ampliación de la gama de robots soportados. En esta versión nos hemos centrado en drones y en Turtlebots, pero existe una completa gama de robots de diferentes características a las que podríamos dar soporte. Uno podría ser el robot humanoide Pepper, muy popular actualmente. Esto supondría la generación de bloques para las necesidades específicas de estos robots.\\

De cara a su usabilidad, se podría ampliar el repertorio de tutoriales y guías didácticas. Enfocándolas al aprendizaje de la programación de robots sofisticados para usuarios con un perfil técnico bajo. Esto podríamos acompañarlo con entornos simulados que se descarguen junto con nuestra herramienta, entornos con características suficientes como para poder programar la resolución de desafíos de diversas complejidades, desde Scratch en conjunción con nuestra herramienta.\\