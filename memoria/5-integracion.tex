integración
Fases:

se recorre un gran camino desde la idea inicial de la herramienta hasta su ejecución final, tanto en tiempo como en desarrollos. 

Vamos a desglosar los diferenetes tramos por los que hemos pasado hasta cerrar una versión standalone y estable de la herramienta.
0-Estado inicial
Partimos de una versión parcialmente independiente de la herramienta ya propuesta por Raúl Perula desarrollada para Gsoc[enlace git-bibliografia].
Como ya hemos hablado anteriormente, podría dividirse la herramienta en dos módulos, el módulo de traducción y generación de código, y la aplicación de este
código sobre robótica.
En esta fase nos encontramos con la parte de traducción y generación funcionando de forma relativamente aislada del resto, 
su ejecución y funcionamiento es totalmente aislada a cualquier dependencia externa ya que se empaqueta en un paquete ROS, con todo lo 
necesario para su ejecución. La dependencia la encontramos a la hora de ejecutar el código generado sobre el robot real o simulado.
Este código está fuertemente acoplado a la plataforma JdeRobot, necesitando de su instalación completa ya que usa muchas de sus librerías.


\section{Integración con JdeRobot}
\label{sec:integracion-con-jderobot}
En esta iteracion buscamos la total integracion de la herramienta en JdeRobot. Trabajando directamente con el equipo de desarrollo de JdeRobot 
agregamos las librerías externas al framework e introducimos la herramienta.

\section{Dependencias de la herramienta}
\label{sec:dependencias}

Dependencias

\section{Creación de paquete ROS Independiente}
\label{sec:paquete-ros}

Una vez introducida en la suit de programación buscamos todas y cada una de las dependencias, se refactoriza el código para depender de las menos librerías posibles 
de JdeRobot y facilitar la futura migración de la herramienta a una versión complatemente independiente.

--EXPLICAR COMO FUNCIONA UN PAQUETE ROS

--SUBID A REPOS OFICIALES PASANDO CALIDAD DE CÓDIGO

--CREACIÓN DE PAQUETES PIP

--- QUE ES UN PAQUETE PIP


\section{Documentación de la herramienta}
\label{sec:documentacion}

--documentacion en ros

--documentacion en git

--documenatacion en pagina d jderobot

--apoyo de videotutoriales
