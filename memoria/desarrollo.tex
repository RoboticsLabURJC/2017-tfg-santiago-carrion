\section{Diseño}
\label{sec:diseno}

Diseño

\section{kurt}
\label{sec:kurt}

Kurt es una biblioteca de Python que permite la manipulación compleja de proyectos Scratch (archivos .sb) a través de simples comandos de Python. Incluye un compilador y descompilador, que permite que un proyecto se cargue en un conjunto de objetos de Python, y un compilador que permite el empaquetamiento de un conjunto de scripts de imágenes / texto en proyectos.

\section{Traduccion con kurt}
\label{sec:traduccion}

traduccion con kurt

\section{Desarrollo de bloques}
\label{sec:desarrollo-de-bloques}


Scratch facilita el uso de bloques personalizados mediante extensiones externas, estas  extensiones externas a la aplicación se definen mediante el uso de ficheros JSON, aunque por convención, en Scratch tentrán extensión .s2e .

Este tipo de ficheros se creó para la comunicación mediante HTTP de bloques con aplicaciones auxiliares, por ejemplo algún tipo de hardware.
Un \textit{AppHelper} se ejecuta en segundo plano, lista para ser utilizada por los proyectos de Scratch que usan esa extensión.
Cada extensión tiene un número de puerto único. Scratch busca la aplicación de ayuda en el número de puerto dado en la máquina local.
Se comunica con el \textit{AppHelper} utilizando el protocolo HTTP, la aplicación envía comandos al \textit{AppHelper} y este envía los valores del sensor y la información de estado a Scratch a través de solicitudes HTTP GET. Dado que el protocolo es HTTP estándar, cualquier navegador se puede usar para probar y depurar aplicaciones de ayuda.

Pero nostros no vamos a utilzar esta funcionalidad, únicamente nos ayudamos de éste documento .s2e para definir nuestra extensión y pueda ser usada desde el IDE offline de scratch.

\subsection{Bloques genericos}


bloques genericos
\subsection{Bloques de drone}
bloques drone
\subsection{robots con ruedas}
bloques robots con ruedas
