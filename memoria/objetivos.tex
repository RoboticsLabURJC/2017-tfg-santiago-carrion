Tras haber expuesto las motivaciones y el contexto en el que se engloba este proyecto de fin de grado, en este capítulo daremos una explicación más detallada del problema que se
intenta resolver y los pasos dados para conseguirlo.


\section{Descripción del problema}
\label{sec:descripcion del problema}

En la actualidad la robótoca está en nuestro alrededor en todo momento, debido a las nuevas tecnologías y al abaratamiento de costes se encuentra en plen auge. A pesar de que convivimos a diario con ella por lo general sigue siendo un enigma para la mayoría de las personas.

No es menos cierto que existe una enorme complejidad que requiere de un alto conocimiento de las tecnoligías implicadas tras la configuración de un robot, y es esta barrera la que queremos eliminar.

Con este proyecto buscamos el acercamiento de la robótica a un público con escasos conocimientos técnicos, simplificando al máximo toda la complejidad que existe tras la programación de un robot, hasta el punto que sea usada por niños para el aprendizaje, haciendo el mundo de la robótica más cercano y más atractivo a ojos de aquellos que serán el futuro de ésta.

Esto lo conseguimos creando un nexo entre un lenguaje de programación visual mediante bloques, para esto hacemos uso de \textit{Scratch}, y la programación de robots en Python. Con \textit{Scratch4Robots} conseguimos que partiendo de un lenguaje fácil e intuitivo, basado en el apilamiento de bloques funcionales de código, la traducción a código Python completamente funcional. Creando unos bloques con funcionalidad diriguida a robots en especifico conseguimos que esta traducción pueda ser aplicada a robots.    

\section{Requisitos}
\label{sec:requisitos}

Para cumplir los objetivos marcados de forma satisfactoria, debemos además satisfacer
los siguientes requisitos:

\begin{itemize}
\item El desarrollo deberá ser autocontenido en lo que sea posible, esto quiere decir que todas las librerías y dependencias de nuestra herramienta deberán estar contenidas en su interior. La programación se realizará en el
lenguaje Python.
\item El software desarrollado debrá ser compatible con Ubuntu 16.04, ROS-kinetic y Scratch 2.0 serán los únicos elementos indispensables para el correcto funcionamiento de nuestra herramienta.
\item Todos los componentes desarrollados deberán ser compatibles tanto trabajando en entornos simulados, como usando robots reales, esto se consigue usando todas las abstracciones de las que nos provee JdeRobots, totalmente testadas en robots reales.
\end{itemize}



\section{Metdología de trabajo}
\label{sec:metodologia}

Metodología

\section{Plan de trabajo}
\label{sec:plan}

Simplificaremos la labor a desarrollar en varios puntos:
\begin{itemize}
\item Familiarización con el entorno software: JdeRobot es la plataforma de desarrollo principal utilizada en la mayoría de proyectos realizados en el departamento de robótica
de la URJC. El objetivo principal de esta fase es aprender a utilizar este software, sus
componentes y sus drivers para más adelante utilizarlos como parte de nuestro proyecto.
Como parte del aprendizaje, se desarrolla una herramienta externa al core principal pero que usa de librerías y recurosos de ella.
\item Estudio de KURT: Librería que nos permite obtener toda la información necesaria de un proyecto Scratch, conocimiento de su API y su funcionamiento interno para saber qué podemos llegar a obtener y como usar esa información para proporcionar una traducción robusta al lenguaje Python.
\item Desarrollo de funcionalidades: Aumentamos el número de bloques robóticos própios que podremos usar en Scratch, esto es desarrollar la lógica detras de cada bloque, todo programado en python y la integración de estos bloques con Scratch. Dividimos los bloques entre aptos para drones y robots con ruedas. Estos bloques deben tener una funcionalidad muy específica y funcionar en armonia con el resto, tanto los propios de la aplicación Scratch como los nuevos bloques generados por nosotros propios de aplicaciones robóticas.
\item Facilitar el uso de la herramienta: Haciendo el uso de la herramienta lo más intuitiva posible, mejorando scripts de lanzamiento y de generación de código. Creando ejemplos autocontenidos para una rapida demostración de la potencia de la herramienta. Generando tutoriales tanto escritos como con video para evitar confusiones.
\item Integración: La parte sin duda más compleja del desarrollo ya que se busca que nuestra aplicación sea facilmente instalable en cualquier entorno y con la mayor facilidad posible, únicamente necesitando las dependencias que hemos comentado con anterioridad. Teniendo en cuenta que debe ser facilmente usada por personas con pocos conocimientos técnicos.


\end{itemize}
